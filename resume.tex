% resume.tex
%
% (c) 2002 Matthew Boedicker <mboedick@mboedick.org> (original author) http://mboedick.org
% (c) 2003-2007 David J. Grant <davidgrant-at-gmail.com> http://www.davidgrant.ca
%
% This work is licensed under the Creative Commons Attribution-ShareAlike 3.0 Unported License. To view a copy of this license, visit http://creativecommons.org/licenses/by-sa/3.0/ or send a letter to Creative Commons, 171 Second Street, Suite 300, San Francisco, California, 94105, USA.

\documentclass[letterpaper,11pt]{article}

%-----------------------------------------------------------
%Margin setup

\setlength{\voffset}{0.1in}
\setlength{\paperwidth}{8.5in}
\setlength{\paperheight}{11in}
\setlength{\headheight}{0in}
\setlength{\headsep}{0in}
\setlength{\textheight}{11in}
\setlength{\textheight}{9.5in}
\setlength{\topmargin}{-0.25in}
\setlength{\textwidth}{7in}
\setlength{\topskip}{0in}
\setlength{\oddsidemargin}{-0.25in}
\setlength{\evensidemargin}{-0.25in}
%-----------------------------------------------------------
%\usepackage{fullpage}
\usepackage{hyperref}
%\textheight=9.0in
\pagestyle{empty}
\raggedbottom
\raggedright
\setlength{\tabcolsep}{0in}

%-----------------------------------------------------------
%Custom commands
\newcommand{\resitem}[1]{\item #1 \vspace{-2pt}}
\newcommand{\resheading}[1]{\vspace{10pt} \Large \textbf{#1} \normalsize}
\newcommand{\ressubheading}[4]{
\begin{tabular*}{6.5in}{l@{\extracolsep{\fill}}r}
		\large \textbf{#1} \normalsize & #2 \\
		\textit{#3} & \textit{#4} \\
\end{tabular*}\vspace{-5pt}}
%-----------------------------------------------------------


\begin{document}

\begin{tabular*}{7in}{l@{\extracolsep{\fill}}r}
\textbf{\Large Matthew Boler}  & 770-855-5529\\
\#501 Webster Road, Lot 315 & meb0054@auburn.edu \\
Auburn, AL 36832 \\ %& \url{https://mattboler.github.io/}\\
\end{tabular*}
\\

\rule{\textwidth}{0.4pt}

\resheading{Education}

\begin{itemize}

\item
	\ressubheading{Auburn University}{Auburn, AL}{Ph.D., Mechanical Engineering}{Anticipated May 2025}
	\begin{itemize}
		\resitem{Research Topic: Robust inertial navigation and optimal control}
	\end{itemize}

\item
	\ressubheading{Auburn University}{Auburn, AL}{M.S., Mechanical Engineering}{May 2022}
	\begin{itemize}
		\resitem{Thesis: "Observability-Informed Measurement Validation for Visual-Inertial Navigation"}
	\end{itemize}

\item
	\ressubheading{Auburn University}{Auburn, AL}{B.S. Mechanical Engineering, Computer Science Minor}{May 2019}

\end{itemize}

\begin{description}
	\item[] \textit{Publications available upon request}
\end{description}

\resheading{Experience}
\begin{itemize}

\item
	\ressubheading{GPS and Vehicle Dynamics Laboratory}{Auburn, AL}{Graduate Research Assistant}{2019 - Current}\\
	\vspace{10pt}

	\large \textbf{RSSI-Aided Navigation} \normalsize
	\vspace{-5pt}
	\begin{itemize}
		\item Implemented an error-state marginalized particle filter to aid an inertial navigation system with signal anomaly maps.
		\item Developed nonparametric mapping methods using gaussian processes to update anomaly maps.
		\item Analyzed sensitivity of navigators to map, sensor, and initialization errors using monte-carlo simulations.
	\end{itemize}

	\large \textbf{Multispectral Visual Navigation} \normalsize
	\vspace{-5pt}
	\begin{itemize}
		\resitem{Developed a multi-state constraint Kalman filter (MSCKF) with holonomic constraints and online extrinsic calibration for GPS-denied infrared+INS ground vehicle navigation.}
		\resitem{Designed a full-smoothing visual-inertial SLAM system using ISAM2 and a novel geometric validation module for robust feature initialization.}
		\resitem{Reduced sensitivity of visual SLAM systems to dynamic environments by adaptively segmenting static and dynamic image regions using YOLO and monitoring feature behavior.}
	\end{itemize}

	\large \textbf{Autonomous Tiger Racing} \normalsize
	\vspace{-5pt}
	\begin{itemize}
		\resitem{Developed a robust ground-removal algorithm for LIDAR obstacle detection to handle large bank angles using a smoothed height-variance map in ROS and PCL.}
		\resitem{Developed a lighweight path planning node in ROS to generate optimal trajectories at 200Hz.}
	\end{itemize}

\item
	\ressubheading{Sandia National Laboratories}{Albuquerque, NM}{Intern - GNC, Autonomy}{2020, 2022}
	\begin{itemize}
		\item Developed a robust real-time lidar odometry for edge platforms using adaptive nonlinear smoothing
		\item Implemented modified Fourier-Mellin, SIFT, and other algorithms to improve registration performance between visual-spectrum and hyperspectral images.
	\end{itemize}

\end{itemize}

\resheading{Skills}

\begin{description}
\item[Languages:]
C++, Python, Matlab, Julia
\item[Software:]
Git, \LaTeX, Docker, Robot Operating System (ROS) 1 and 2
\end{description}

\end{document}
